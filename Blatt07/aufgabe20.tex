\section{Aufgabe 20}
\subsection{Local Bubble}
Die Lokale Blase ist ein staubfreies Gebiet, welches unsere Sonne gerade durchwandert. Sie misst ca. einen Druchmesser von 300lj. In diesem Gebiet ist kaum Staub vorhanden und auch nur ganz wenig Wasserstoffatome (0.05 Atome pro $cm^3$). das heiße diffuse Gas in der lokalen Blase strahlt im Gamma-Bereich. Es wird vermutet das eine Supernova diese Blase erzeugt hat. Als einer der wahrscheinlichsten Kandidaten gilt Geminga, jetzt ein Neutronenstern. Die Form der Blase ähnelt wahrscheinlich einer Sanduhr. Aber da leider die UV und die Röntgenstrahl werte nicht zusammenpassen ist das noch nicht sicher. 
\subsection{Local Chimney}
Wahrscheinlich ist unsere lokale Blase ein Teil eines sogenannten hot chimney.
\subsection{Local Fluff}
Die Lokale Flocke (en. Local Fluff) wird auch die lokale interstellare Wolke genannt. Sie befindet sich innerhalb der lokalen Blase und hat einen Durchmesser von ca. 30lj. In der Flocke ist die Dichte ein bisschen höher als sonst in der lokalen Blase (0.1 Atom pro $cm^3$).
\subsection{Gould's Belt}
Momentan vermutet man das der Gürtel durch eine Kollision von dunkler Materie und der lokalen molekularen Wolke vor rund 30 Millionen Jahren entstanden ist. Dieser Ring um das galaktische Zentrum ist nicht ganz vollständig und ca. 16 bis 20 Grad geneigt. Er hat einen Durchmesser von ca. 2000lj.
\subsection{Sternassoziation}
Wenn 5 bis 100 junge in einer losen Gruppierung sind und diese ähnliche physikalische Eigenschaften habe spricht man von einer Sternassoziation. Wenn diese Sterne auch noch eine ähnliche Bewegungsrichtung haben kann man die Gruppierung auch zu Sternströmungen zählen. Sternassoziation bestehen nicht sehr lange. Die Sterne sind im allgemeinen vom selbe Sternentstehungsgebiet. Die Sternassoziation ist auch die loseste Form eines offenen Sternhaufens, da diese am wenigsten gebunden sind (gravitativ nicht gebunden). Man kann sich nach der physikalischen Eigenschaften der Mitgliedssterne kategorisieren. 
So ist die OB-Assoziation eine die vor allem aus O und B Sternen besteht. Hingegen die T-Assoziation besteht aus veränderlichen T-Tauri Sterne (Sterne die noch nicht die Hauptreihe erreicht haben und noch kein inneres Gleichgewicht gefunden haben).
Gruppierungen von Sterne die ca. 1 Millionen Jahre alt sind und in einem Reflexionnebel eingebettet sind nennt man R-Assoziation.
