\section{Aufgabe 20}
\subsection{Local Bubble}
300lj
teil vom local chimney
\subsection{Local Chimney}
\subsection{Local Fluff}
Die Lokale Flocke (en. Local Fluff) wird auch die lokale interstellare Wolke genannt. Sie befindet sich innerhalb der lokalen Blase und hat einen Durchmesser von ca. 30lj.
\subsection{Gould's Belt}
Sternentsteunggebiet 2000lj
\subsection{Sternassoziation}
Wenn 5 bis 100 junge in einer losen Gruppierung sind und diese ähnliche physikalische Eigenschaften habe spricht man von einer Sternassoziation. Wenn diese Sterne auch noch eine ähnliche Bewegungsrichtung haben kann man die Gruppierung auch zu Sternströmungen zählen. Sternassoziation bestehen nicht sehr lange. Die Sterne sind im allgemeinen vom selbe Sternentstehungsgebiet. Die Sternassoziation ist auch die loseste Form eines offenen Sternhaufens, da diese am wenigsten gebunden sind (gravitativ nicht gebunden). Man kann sich nach der physikalischen Eigenschaften der Mitgliedssterne kategorisieren. 
So ist die OB-Assoziation eine die vor allem aus O und B Sternen besteht. Hingegen die T-Assoziation besteht aus veränderlichen T-Tauri Sterne (Sterne die noch nicht die Hauptreihe erreicht haben und noch kein inneres Gleichgewicht gefunden haben).
Gruppierungen von Sterne die ca. 1 Millionen Jahre alt sind und in einem Reflexionnebel eingebettet sind nennt man R-Assoziation.
