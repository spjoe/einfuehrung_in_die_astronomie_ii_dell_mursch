\section{Aufgabe 28}

\begin{center}
\begin{tabular}{|l|l|l|l|l|l|l|l|}
\hline
\(Name\) & $n(pc^{-3})$ & $v(km/s)$ & $N$ & $D(kpc)$ & $\tau_{cr}(s)$ & $\tau_{rel}(s)$ & $\tau_{rel}/\tau_{cr}$ \\
\hline
Offene Sternhaufen & 1 & 1 & $10^5$ & 0.04 & $1,23 \cdot 10^{15}$  & $5,31 \cdot 10^{17}$  & $430$\\			
\hline
Globular Cluster & $10^2$ & 10 & $10^6$ & $10^{-2}$ & 3,09E+13 & 1,47E+19 & 4,78E+05\\ 				

\hline
Galaxienscheibe & $0.1$ & $20$ & $10^{10}$ & $10$ & $1,54E+16$ & $1,18E+23$ &  $7,64E+06$ \\ 				

\hline
Galaxienhalo & $10^{-2}$ & $200$ & $10^{11}$ & $50$ & $7,71E+15$ &  $1,18E+27$ & $1,53E+11$ \\ 				

\hline
Galaxienhaufen & $10^{-12}$ & $10^3$ & $10^3$ & $10^3$ & $3,09E+16$  & $5,31E+18$ &  $1,72E+02$\\ 		
	

\hline
\end{tabular}\\
\end{center}
Zu Veranschaulichung wie man zu den Zeiten kommt werde ich es anhand des Offenen Sternhaufens illustrieren. 
Die Crossing Time ist ganz leicht zu bestimmen.\\
$\tau_{cr} = D/v$\\
Da die Einheiten noch nicht zusammen passen rechne ich die kpc in km um. Mit dem Faktor $3,08567758129*10^{16} (\frac{km}{kpc})$ kann man dies erreichen. Somit ergibt sich eine Zeitspanne von $1,23E+15 (s)$. Mit dem Faktor $\frac{1}{3,154*10^{13}} (\frac{Myr}{s})$ kann man das Ergebnis in Myr umrechnen.
Somit sind $3,91E+01 (Myr)$ gleich $1,23E+15 (s)$.\\
Für die Berechnung von der Relaxationszeit muss man ein bisschen mehr beachten. Die Formel ist:\\
$\tau_{rel} = v^3/\pi n G^2 m^2$\\
Hier muss man wieder auf die Einheiten aufpassen und das richtige m herausfinden. Bei offenen Sternhaufen sind die meisten Sterne jung und schwer somit ist m groß. Aber m hängt auch vom alter des offenen Sternhaufens ab, da dieser sich sehr schnell zerstreut. (Fraglich wie sinnvoll es ist bei offenen Sternhaufen die Relaxationszeit zu bestimmen.) Ich wählte das m bei einem offenen Sternhaufen eine Sonnenmasse, bei Globular cluster, Galaxienscheibe, Galaxienhalo 0.6 Sonnenmassen. Und beim Galaxienhaufen $10^{10}$ Sonnenmassen.
