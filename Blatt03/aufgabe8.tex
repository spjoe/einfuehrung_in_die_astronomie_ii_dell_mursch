\section{Aufgabe 8}
Für die Jeans-Masse \(M_J\) gilt:
\begin{equation}
M_J = \left(\frac{5~k~T}{G~\mu~m_H}\right)^{3/2} \cdot \left(\frac{3}{4~\pi~n~m_H}\right)^{1/2}
\end{equation}
\\
\(k\dots\) Boltzmann-Konstante (\(k = 1.381 \times 10^{-23}~J~K^{-1}\))\\
\(G\dots\) Gravitationskonstante (\(G = 6.674 \times 10^{-11}~m^3~kg^{-1}~s^{-2}\))\\
\(m_H\dots\) Masse des Wasserstoff-Atoms (\(m_H = 1.674 \times 10^-27~kg\))\\
\(T\dots\) Temperatur\\
\(\mu\dots\) Mittleres Molekulargewicht\\
\(n\dots\) Teilchendichte

\subsection{Wolke aus molekularem Gas}
\(\mu = 2.4\)\\
\(n = 10~cm^{-3} = 10^7~m^{-3}\)\\
\(T = 100~K\)
\\
\begin{equation}
(1) \implies M_{J,2.1} = 1.463 \times 10^{44}~kg
\end{equation}

\subsection{Wolke aus reinem, atomarem Wasserstoff}
\(\mu = 1\)\\
\(n = 1~cm^{-3} = 10^6~m^{-3}\)\\
\(T = 10^4~K\)
\\
\begin{equation}
(1) \implies M_{J,2.2} = 5.44 \times 10^{48}~kg
\end{equation}

\subsection{Virial-Masse}
Die Virial-Masse ist die Masse einer Gaswolke die sich im Virial-Gleichgewicht befindet.
\begin{align}
2~E_{th} &= -E_{pot}\nonumber\\
2~(\frac{3}{2}~M_V~\frac{k~T}{\mu~m_H}) &= -(-\frac{3}{5}~\frac{G~M_V^2}{R})\\
M_V &= \frac{5~k~T~R}{G~\mu~m_H}\nonumber
\end{align}
\\
\(E_{th}\dots\) Thermische Energie\\
\(E_{pot}\dots\) Potentielle Energie\\
\(R\dots\) Radius der Gaswolke\\
\(M_V\dots\) Virial-Masse\\
\\
Wenn eine Gaswolke eine grö\ss ere Masse als die Jeans-Masse hat wird sie instabil wird und beginnt sich zu kontrahieren. 
Also wenn \(|E_{pot}| > 2~E_{th}\) wird.\\
Daher stimmen die Virial- und Jeans-Masse bei einer Gaswolke überein.
