\section{Aufgabe 7}
10 Mohlekühle
\begin{enumerate}
\item $^{13}C$
\item CO
\item $CH_{3}OH$ (Methanol) astrochemistry.net
\item $NH_2CH_2COOH$ (Glycine) astrochemistry.net
\item $H_2CCC$ (Propadienylidene) astrochemistry.net
\item $CH_3OCH_3$ Dimethyl ether astrochemistry.net
\item $FeO$	Iron Oxide astrochemistry.net
\item $KCl$	Potassium chloride astrochemistry.net
\item $C_2H$ Ethynyl radical astrochemistry.net
\item $C_2$ Dicarbide astrochemistry.net
\end{enumerate}
Höchste Rotverschiebung \\
IOK-1 \cite{iye2006galaxy} hat ein z von 6,96. Das bedeutet, dass die Galaxie ca. 13 Milliarden Jahre alt ist.
\subsection{SOFIA}
Im SOFIA-Teleskop könne verschieden Instrumente eingebaut werden. 
Das SAFIRE Instrument kann für Galaxien mit hoher Rotverschiebung beobachten.
FIFI LS ist ein Instrument was unter anderem Wenig-Metallische kleine Galaxien im frühen Universum untersuchen kann.
\subsection{HERSCHEL}
60 bis 670 Mikrometer Wellenlänge
Entstehung und Entwicklung von Galaxien, insbesondere entfernter junger Galaxien, die aufgrund ihres Staubgehalts hauptsächlich im fernen Infrarot ausstrahlen. \\
Physik und Chemie der interstellaren Materie.
\subsection{ALMA}
Infrarotgalaxien im frühen Universum. Aufzeichnen von Staub was zu einer Galaxie wird im frühen Universum, mit bis zu z=10.


