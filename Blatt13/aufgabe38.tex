\section{Aufgabe 38}
\subsection{Internet}
Laut der englischen Wikipedia wissen wir von 17 Zwerggalaxien, die Entfernungen zur Milchstrasse variieren von 0.08 Mpc bis 0.5 Mpc

\begin{tabular}{|l|l|l|l|l|l|}
\hline Name & Durchmesser & [Fe/H] & Masse & Leuchtkraft & Distanz \\ 
\hline Milchstrasse & $10^5 ly$ & - & $5.8 \times 10^{11} M_\odot$ & $-20.8 M$ &  0 kpc\\ 
\hline Sgr I Zwerggalaxie & $10^4 ly$ & $-1.6 \pm 0.1$ & - & $-12.67 M$ & 8 kpc\\ 
\hline große Magellanschen Wolke & $1.4 \times 10^4 ly$ & - & $10^{10} M_\odot$ & $-17.93 M$  & 50 kpc  \\
\hline kleine Magellanschen Wolke & $7 \times 10^3 ly$ &  - & $7 \times 10^9 M_\odot$  & $-16.35 M$ & 63 kpc \\  

\hline 
\end{tabular} 



\subsection{Puplikation}
In der Publikation \cite{mateo1998dwarf} werden 14 Zwerggalaxien der Milchstraße zugeordnet. Diese liegen in einem Abstand von 24 kpc bis 445 kpc.
%In dem etwas neuerem Paper \cite{di2006dwarf} sind von xx Zwerggalaxien die rede und diese sind in einem Abstand von xx bis xx kpc.


\begin{tabular}{|l|l|l|l|l|l|}
\hline Name & Durchmesser & Metallizität & Masse & $M_V$ & Distanz \\ 
\hline Milchstrasse &  &  &  &  &  \\ 
\hline Sgr I Zwerggalaxie &  &  &  & $-13.4 M$ &  \\ 
\hline große Magellanschen Wolke &  &  &  &  &  \\ 
\hline kleine Magellanschen Wolke &  &  &  &  &  \\ 

\hline 
\end{tabular} 
