\section{Aufgabe 39}
Gravitationskraft die Planethälften zusammen hält.\\
$F_g=\frac{m/2 * m/2}{r^2} G = \frac{G \frac{m}{4}}{r^2} $\\

Gezeitenkraft die Planet auseinderreißt:\\
Zu einem haben wir die Zentripetalkraft(zp). In diesem Fall ist das die Gravitationskraft zwischen dem Planten und dem Zentralgestirn. Diese ist jedoch auf der äußeren(a) Hälfte schwächer als auf der inneren(i). \\
$F_{azp} = \frac{M\frac{m}{2}}{(D+\frac{r}{2})^2}G $\\
$F_{izp} = \frac{M\frac{m}{2}}{(D-\frac{r}{2})^2}G $\\
Als gegen Kraft spürt die Erde die Zentrifugalkraft(zf). Die ist für den äußeren Halbplaneten stärker zu spüren.\\
$F_{azf} = \frac{m}{2}\omega^2(D + \frac{r}{2})$\\
$F_{izf} = \frac{m}{2}\omega^2(D - \frac{r}{2})$\\

Die Winkelgeschwindigkeit des gesamten Planeten kann in Abhängigkeit gebracht werden mit der Masse des Zentralkörpers und dem Abstand zu diesem.\\
$\omega^2 = \frac{G M}{D^3}$\\
Nun ersetzen wir omega und vereinfachen.\\
$F_{azf} = \frac{m}{2}\frac{G M}{D^3}(D + \frac{r}{2})$\\
$F_{izf} = \frac{m}{2}\frac{G M}{D^3}(D - \frac{r}{2})$\\

Für Planeten die nicht zerreißen soll, muss folgendes gelten:\\
$F_g > F_{azf} - F_{azp}$\\
$F_g > F_{izp} - F_{izf}$\\

Die Grenze liegt folglich bei $F_g = F_{azf} - F_{azp}$:\\


