\section{Aufgabe 13}
Initial Mass Function (IMF):
\begin{equation}
dN(M) \symbol{126} M^\gamma dM
\end{equation}
\begin{equation}
\int_{m_i}^{m_u} \frac{dN(M)}{dM} \, dM=1
\end{equation}
\begin{equation}
\int_{m_i}^{m_u} M^\gamma \, dM=1
\end{equation}
\subsection{a) Massenanteil massenreicher Sterne}
Salpeter-Faktor:\\
Normalisierungsfaktor Berechnen, der Bereich [0.1,100] ist 1
\begin{equation}
A=\frac{1}{\int _{0.1}^{100}M^{-2.35}*MdM}
\end{equation}
Anteil für [10,100] berechnen.
12\%
\begin{equation}
\int _{10}^{100}A M^{-2.35}*MdM = 0.12 ~ \widehat{=} ~ 12\%
\end{equation}
Bei Kroupa:\\
Normalisierungsfaktor Berechnen, der Bereich [0.1,100] ist 1
\begin{equation}
B=\frac{1}{\int _{0.1}^{100}M^{-2.7}*MdM}
\end{equation}
Anteil für [10,100] berechnen.
3\%
\begin{equation}
\int _{10}^{100}B M^{-2.7}*MdM = 0.03 ~ \widehat{=} ~ 3\%
\end{equation}
\subsection{b) Sternenzahl massereicher Sterne}

Salpeter-Faktor:\\
Normalisierungsfaktor Berechnen, der Bereich [0.1,100] ist 1
\begin{equation}
B = \frac{1}{\int_{0.1}^{100} M^{-2.35} \, dM} = \frac{1}{16.5816}
\end{equation}
Anteil für [10,100] berechnen.
\begin{equation}
\int_{10}^{100} B M^{-2.35} \, dM = 0.00191 ~ \widehat{=} ~ 0.19\%
\end{equation}
Bei Kroupa:\\
Normalisierungsfaktor Berechnen, der Bereich [0.1,100] ist 1
\begin{equation}
A = \frac{1}{\int_{0.1}^{100} M^{-2.7} \, dM} = \frac{1}{29.48}
\end{equation}
Anteil für [10,100] berechnen.
\begin{equation}
\int_{10}^{100} A M^{-2.7} \, dM = 0.00039 ~ \widehat{=} ~ 0.04\%
\end{equation}
\subsection{c) mittlere Masse massereicher Sterne}
\begin{equation}
M^\gamma * M = 0, M > 10, M < 100
\end{equation}
