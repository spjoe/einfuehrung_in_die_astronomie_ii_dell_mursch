\section{Aufgabe 14}
\subsection{Infrared Space Observatory (ISO)}
Das ISO startet am 17.11.1995 mit der Ariane 44P, und nach 28 Monaten und 22 Tagen Laufzeit hörte die Mission auf. Wie der Name schon sagt untersucht das ISO die infraroten Welllängen und zwar von 2.5 bis 240 Mikrometer mit Photometrie und von 2.5 bis 196.8 Mikrometer mit Spektroskopie.\\
Der Satellit besteht als zwei Modulen, das Lastmodul(Payload module) und dem Servicemodul(service module). Im Lastmodul ist das Kryostat (Kühlgerät)m das Teleskop und die 4 wissenschaftlichen Instrumente angebracht. Im Servicemodul ist die Energiezufuhr, die Wärmekontrolle, die Steuerung und die Telekommunikation untergebracht.\\
Es gibt vier Instrumente am Satelliten, eine ist die ISOCAM, dass ist eine hochauflösende Kamera die eine Wellenlänge von 2.5 bis 17 Mikrometer detektieren kann. Dazu werden zwei Detektoren gebraucht.\\
Ein weiteres Instrument ist das ISOPHOT, dieses Instrument wurde entworfen um auch sehr kalte Objekt wie z.b. interstellaren Staubwolken aufzunehmen. Die Wellenlänge von diesem gerät geht von 2.4 Mikrometer bis 240 Mikrometer.\\
Das Short Wave Spectrometer (SWS), ist ein Spektrometer das von einer Wellenlänge von 2.4 bis 45 Mikrometer arbeitet. Dadurch kann man die chemische Zusammensetzung, die Dicht und Temperatur von Universum gut messen.\\
Im Gegensatz das Long Wave Spectrometer (LWS), spektroskopiert über einen Bereich von 45 bis 196.8 Mikrometer Wellenlänge. Dadurch kann es viel kühler Objekte als das SWS studieren. Z.B. kalte Staubwolken.\\
Das ISO hat im ISM chemische Prozesse, Moleküle im besonderen Eisenverbindungen beobachtet.
\subsection{Spitzer Space Telescope (SST)}
Dieses Teleskop wurde am 25.08 2003 mit einer Delta II gestartet. Das SST arbeitet bis heute, es umkreist die Sonne innerhalb eines Jahres. Die Wellenlänge die es untersucht gehen von 3 bis zu 180 Mikrometer. \\
Es sind 3 Instrumente zu Detektieren angebracht. Das IRAC, IRS und MIPS sind in der MIC (Multiple Instrument Chamber) untergebracht. Das SST ist ca. 4 Meter groß und 865 kg schwer. Hat ein Solarmodul und eine Schutz fürs Solarmodul. Das Teleskop und die MIC und das Kyrostat sind in einem Modul und durch ein Schilt geschützt. \\
IRAC (engl. Infrared Array Camera) ist eine INfrarotkamera, dies kann gleichzeitig vier Kanäle aufnehmen. Und zwar mit den wellenlängen 2.6 Mikrometer, 4.5 Mikrometer, 5.8 Mikrometer und 8 Mikrometer.\\
Das IRS (engl. Infrared Spectrograph) ist ein Spektrometer von 5.3 bis 37 Mikrometer.\\
Die MIPS (engl. Mulitband Imaging Photometer for Spitzer) setzt sich aus drei Detektor-Feldern zusammen, die im fernen Infrarotbereich liegen. Von 24 bis 160 Mikrometer.\\
Weil die SST im Infrarotbereich arbeite konnte man das Innere der Milchstrasse gut aufnehmen, den im sichtbaren Bereich ist das Zentrum der Milchstrasse durch ISM verdeckt.
\subsection{Stratospheric Observatory For Infrared Astronomy (SOFIA)}
\subsection{Herschel Space Observatory (Herschel)}

