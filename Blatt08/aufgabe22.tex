\section{Aufgabe 22}
Intensität-Magnituden Zusammenhang:\\
$log(\frac{I_1}{I_2}) = -0.4(m_1 - m_2)$\\
$I_2 = 10^{0.4(m_1-m_2)}I_1$ \\
Intensität-Masse Verhältniss:\\
$L_v = M_*^3$\\
Sternenmasse Verteilung:\\
$dN(M_v) = M^\gamma dM$ \\
Intensität-Masse Zusammenhang einsetzen: \\
$dN(L_v) = L_v^{\gamma/3} dL$ \\
Intensität-Magnituden Zusammenhang einsetzen: \\
$dN(m_v) = \left (10^{0.4(4.8 - m)} \right)^{\gamma/3} dm)$ \\
Salpeter: $\gamma = -2.35$\\
$LF  = \int_{m_i}^{m_u} \left (10^{0.4(4.8 - m)} \right)^{\gamma/3} dm$\\

Gefunde Formel:\\
$N(m) = \int_{0}^{\infty} D(\vec r)\Phi\left [ m-5log \frac{|\vec r|}{10pc} -\gamma(\vec r) \right ] \omega r^2 dr$
