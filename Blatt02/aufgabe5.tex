\section{Aufgabe 5}
\subsection{Supernovas in den letzten 1000 Jahren}
\begin{center}
	\begin{tabular}{ l | l | l | l | l | l |}
	\bf{Bezeichnung} & \bf{E. Jahr} & \bf{A. Sonne} & \bf{Rektaszension} & \bf{Deklination} & \bf{SN-Typ} \\
	\hline
	Supernova1006 & $1006^{1}$ & $2,2kpc^3$ & $15h02.8m^1 $&  $-41^\circ57'^1$ & $Ia^2$ \\ \hline
	Supernova1054 & $1054^{2}$ & $6.100ly^1$ & $05h34.5m^1$ & $+22^\circ01'^1$ & $II o. Ib^2$ \\ \hline
	Supernova1181 & $1181^2$ & k.A. & $2h06^1$ & $+64^\circ49'$ & $II o. Ib^2$ \\ \hline
	Supernova1572 & $1572^1$ & $10.000 ly^1$ & $00h25.3m^1$ & $+64^\circ09'^1$ & $Ia^2$ \\ \hline
	Supernova1604 & $1604^1$ & $< 20.000 ly^1$ &  $17h30.6m^1$ & $-21^\circ29'^1$ & $Ib^2$ \\ \hline
	Cassiopeia A & $1680^3 echo 2008^3$ & $11.000 ly^3$ & $23h23m24s^3$ & $+58^\circ48'54''^3$ & $IIb^3$ \\ \hline
%	S Andromedae & 1885 & & & & \\ \hline
%	SN 1979C & & & & & \\ \hline
	Supernova1987A & $1987^3$ & $51,4kpc^3$ & $05h35m28.3s^3$ & $-69^\circ16'12''^1$ & $II-P^3$ \\ \hline
%	SN 1994D & & & & & \\ \hline
%	SN 2005cs & & & & & \\ \hline
%	SN 2006gy & & & & & \\ \hline
%	SN 2008D & & & & & \\ \hline
	\end{tabular}
\end{center}
1) SEDS
2) Polcaro(2006) \cite{polcaro2006supernovae}
3) wiki
\subsection{Supernova 1987A}
Die Supernova wurde am 24. Februar von zwei Observatorien unabhängig entdeckt.Und zwar von Oscar Duhalde und Ian Shelton in Las Campanas und von Albert Jones in Neuseeland. Diese Supernova ist die Nächste seit der Supernova von 1604. Sie explodierte in einer Entfernung von 51.4kpc. Sehr ungewöhnlich bei dieser Supernova, war das ausbleiben von Neutrinos. \cite{bionta1987observation}
