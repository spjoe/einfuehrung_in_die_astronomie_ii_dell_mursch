\documentclass{scrartcl}
\usepackage[utf8]{inputenc}
\usepackage[T1]{fontenc}
\usepackage[ngerman]{babel}
\usepackage{amsmath}
 
\title{Einführung in die Astronomie II}
\author{Camillo Dell'mour, Rainer Mursch-Radlgruber}
\date{24.03.2010}
\begin{document}
\maketitle
%\tableofcontents
\section{Aufgabe 5}
\subsection{Supernovas in den letzten 1000 Jahren}
\begin{center}
	\begin{tabular}{ l | l | l | l | l |}
	\bf{Bezeichnung} & \bf{Entdeckungs Jahr} & \bf{Abstand zur Sonne} & \bf{Koordinaten} & \bf{SN-Typ} \\
	\hline
	Supernova 185 & & & & \\ \hline
	386 & & & & \\ \hline
	393 & & & & \\ \hline
	Supernova1006 & & & & \\ \hline
	Supernova1054 & & & & \\ \hline
	Supernova1181 & & & & \\ \hline
	Supernova1572 & & & & \\ \hline
	Supernova1604 & & & & \\ \hline
	1680 Cassiopeia A & & & & \\ \hline
	1885 S Andromedae & & & & \\ \hline
	SN 1979C & & & & \\ \hline
	Supernova1987A & & & & \\ \hline
	SN 1994D & & & & \\ \hline
	SN 2005cs & & & & \\ \hline
	SN 2006gy & & & & \\ \hline
	SN 2008D & & & & \\ \hline
	\hline
	\end{tabular}
\end{center}

\subsection{Supernova 1087A}

%\bibliography{literatur}     %BibTeX-Datei literatur.bib
\section{Aufgabe 6}
\end{document}

