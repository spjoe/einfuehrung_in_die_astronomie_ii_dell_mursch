\documentclass{scrartcl}
\usepackage[utf8]{inputenc}
\usepackage[T1]{fontenc}
\usepackage[ngerman]{babel}
\usepackage{amsmath}
 
\title{Einführung in die Astronomie II}
\author{Camillo Dell'mour 0628020, Rainer Mursch-Radlgruber 0828445}
\date{24.03.2010}
\begin{document}
\maketitle
%\tableofcontents
\section{Aufgabe 5}
\subsection{Supernovas in den letzten 1000 Jahren}
\begin{center}
	\begin{tabular}{ l | l | l | l | l | l |}
	\bf{Bezeichnung} & \bf{E. Jahr} & \bf{A. Sonne} & \bf{Rektaszension} & \bf{Deklination} & \bf{SN-Typ} \\
	\hline
	Supernova 185 & & & & & \\ \hline
	386 & & & & & \\ \hline
	393 & & & & & \\ \hline
	Supernova1006 & $1006^{1}$ & $80pc^3$ & $15h02.8m^1 $&  $-41^\circ57'^1$ & $Ia^2$ \\ \hline
	Supernova1054 & $1054^{2}$ & $6.100ly^1$ & $05h34.5m^1$ & $+22^\circ01'^1$ & $II o. Ib^2$ \\ \hline
	Supernova1181 & $1181^2$&  & & & $II o. Ib^2$ \\ \hline
	Supernova1572 & & & & & \\ \hline
	Supernova1604 & & & & & \\ \hline
	Cassiopeia A & 1680 & & & & \\ \hline
	S Andromedae & 1885 & & & & \\ \hline
	SN 1979C & & & & & \\ \hline
	Supernova1987A & & & & & \\ \hline
	SN 1994D & & & & & \\ \hline
	SN 2005cs & & & & & \\ \hline
	SN 2006gy & & & & & \\ \hline
	SN 2008D & & & & & \\ \hline
	\hline
	\end{tabular}
\end{center}
1) SEDS
2) Polcaro(2006) \cite{polcaro2006supernovae}
\subsection{Supernova 1087A}


\section{Aufgabe 6}
\clearpage 
\addcontentsline{toc}{chapter}{\bibname}
\bibliographystyle{bababbrv} 
\bibliography{literatur}     %BibTeX-Datei literatur.bib
\end{document}

