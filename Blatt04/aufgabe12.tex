\section{Aufgabe 12}
Es sollen folgende Werte Berechnet werden $\lambda_J$, $M_J$, $\tau_{cool}$ und $\tau_{ff}$. Mit den folgenden Formeln aus dem Skriptum wird gerechnet.
\begin{equation}{Jeans ~ Laenge ~ }
= \lambda_{J}=\sqrt{\frac{\pi {c_s}^2}{G \rho_0}}
\end{equation}
\begin{equation}{Jeans ~ Masse ~ }
= M_{J}=\frac{4 \pi}{3} \rho_0 \left ( \frac{\lambda_J}{2} \right)^3
\end{equation}
\begin{equation}{Abkuehlzeit ~ }
= \tau_{cool}=\frac{3}{2} \frac{k T}{\Lambda_0 n}
\end{equation}
\begin{equation}{Freie ~ Fallzeit ~ }
= \tau_{ff}=\sqrt{\frac{3 \pi}{32 G \rho}}
\end{equation}
Sonstige Gleichungen und Konstanten die gebraucht werden:
\begin{equation}{Schallgeschwindigkeit ~ }
= c_{s}=\sqrt{\frac{k T}{m}}
\end{equation}
\begin{equation}{Massendicht ~ }
= \rho=m_H \cdot n
\end{equation}
\(k\dots\) Boltzmann-Konstante (\(k = 1.381 \times 10^{-23}~J~K^{-1}\))\\
\(G\dots\) Gravitationskonstante (\(G = 6.674 \times 10^{-11}~m^3~kg^{-1}~s^{-2}\))\\
\(m_H\dots\) Masse des Wasserstoff-Atoms (\(m_H = 1.674 \times 10^-27~kg\))\\
\(\pi\dots\) Pi(\(\pi=3.14\))\\
Legende der Variablen:\\
\(T\dots\) Temperatur\\
\(n\dots\) Teilchendichte\\
\(\rho\dots\) Massendichte\\
\\
Die Berechnung der drei(4) Phasen des McKee-Ostriker Modells \\
CNM Phase: \\
T=80 K, $n = 42 cm^{-3}$, $x=10^{-3}$, $\Lambda_0 = 10^{-26} \frac{erg~cm^3}{s}$\\
\begin{equation}
\rho=1.674 \times 10^-27~kg \cdot 42 cm^{-3} = 7.031 \times 10^{-26} \frac{kg}{cm^3}=7.031 \times 10^{-20} \frac{kg}{m^3}
\end{equation}
\begin{equation}
c_{s}=\sqrt{\frac{ 1.381 \times 10^{-23}~J~K^{-1}\cdot 80 K}{1.674 \times 10^-27~kg}} = 812.4 \frac{m}{s}
\end{equation}
\begin{equation}
\lambda_{J}=\sqrt{\frac{\pi \cdot 812.4^2}{6.674 \times 10^{-11}~m^3~kg^{-1}~s^{-2} \cdot 7.031 \times 10^{-20} \frac{kg}{m^3}}} = 2.33 \times 10^{16} m
\end{equation}
\begin{equation}
M_{J}=\frac{4 \pi}{3} 7.031 \times 10^{-20} \frac{kg}{m^3} \left ( \frac{2.1 \times 10^{19} m}{2} \right)^3 = 4.67 \times 10^{29} kg
\end{equation}
\begin{equation}
\tau_{cool}=\frac{3}{2} \frac{1.381 \times 10^{-16}~erg~K^{-1} \cdot 80 K}{10^{-26} \frac{erg~cm^3}{s} \cdot 42 cm^{-3}} = 3.945 \times 10^{10} s
\end{equation}
\begin{equation}
\tau_{ff}= \sqrt{\frac{3 \pi}{32 \cdot 6.674 \times 10^{-11}~m^3~kg^{-1}~s^{-2} 7.031 \times 10^{-20} \frac{kg}{m^3}}} = 2.51 \times 10^{14} s
\end{equation}
(WNM und) WIM Phase: \\
T=8000 K, $n = 0.25 cm^{-3}$, $x=0.68$, $\Lambda_0 = 10^{-22} \frac{erg~cm^3}{s}$\\
$\rho= 4.185 \times 10^{-22} \frac{kg}{m^3}$, $c_{s}=8123 \frac{m}{s}$, $\lambda_{J}= 9.559 \times 10^{17} m$, $M_{J}= 1.914 \times 10^{32} kg$, $\tau_{cool} = 6.6288 \times 10^{10} s$ und $\tau_{ff} = 3.247 \times 10^{15} s$\\
HIM Phase: \\
$T=4.5 \times 10^{5} K$, $n = 3.5 \times 10^{-3} cm^{-3}$, $x=1.0$, $\Lambda_0 = 10^{-22} \frac{erg~cm^3}{s}$\\
$\rho= 5.859 \times 10^{-24} \frac{kg}{m^3}$, $c_{s}=19267 \frac{m}{s}$, $\lambda_{J}= 1.244 \times 10^{19} m$, $M_{J}= 5.908 \times 10^{33} kg$, $\tau_{cool} = 4.734 \times 10^{12} s$ und $\tau_{ff} = 2.744 \times 10^{16} s$\\
