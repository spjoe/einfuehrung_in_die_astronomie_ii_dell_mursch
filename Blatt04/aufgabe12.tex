\section{Aufgabe 12}
Es sollen folgende Werte Berechnet werden $\lambda_J$, $M_J$, $\tau_{cool}$ und $\tau_{ff}$. Mit den folgenden Formeln aus dem Skriptum wird gerechnet.
\begin{equation}{Jeans ~ Laenge ~ }
= \lambda_{J}=\sqrt{\frac{\pi {c_s}^2}{G \rho_0}}
\end{equation}
\begin{equation}{Jeans ~ Masse ~ }
= M_{J}=\frac{4 \pi}{3} \rho_0 \left ( \frac{\lambda_J}{2} \right)
\end{equation}
\begin{equation}{Abkuehlzeit ~ }
= \tau_{cool}=\frac{3}{2} \frac{k T}{\Lambda_0 n}
\end{equation}
\begin{equation}{Freie ~ Fallzeit ~ }
= \tau_{ff}=\sqrt{\frac{3 \pi}{32 G \rho}}
\end{equation}
Sonstige Gleichungen und Konstanten die gebraucht werden:
\begin{equation}{Schallgeschwindigkeit ~ }
= c_{s}=\sqrt{\frac{\gamma \cdot P}{\rho}}
\end{equation}
\begin{equation}{Massendicht ~ }
= \rho=m_H \cdot n
\end{equation}
\(k\dots\) Boltzmann-Konstante (\(k = 1.381 \times 10^{-23}~J~K^{-1}\))\\
\(G\dots\) Gravitationskonstante (\(G = 6.674 \times 10^{-11}~m^3~kg^{-1}~s^{-2}\))\\
\(m_H\dots\) Masse des Wasserstoff-Atoms (\(m_H = 1.674 \times 10^-27~kg\))\\
\(\pi\dots\) Pi(\(\pi=3.14\))\\
Legende der Variablen:\\
\(T\dots\) Temperatur\\
\(n\dots\) Teilchendichte\\
\(\gamma\dots\) bla\\
\(\rho\dots\) Massendichte\\
\\
Die Berechnung der drei(4) Phasen des McKee-Ostriker Modells \\
CNM Phase: \\
T=80 K, $n = 42 cm^{-3}$, $x=10^{-3}$, $\Lambda_0 = 10^{-26} \frac{erg~cm^3}{s}$\\
\begin{equation}
\rho=1.674 \times 10^-27~kg \cdot 42 cm^{-3} = 7.031 \times 10^{-26} \frac{kg}{cm^3}
\end{equation}
\begin{equation}
c_{s}=\sqrt{\frac{\gamma \cdot P}{\rho}}
\end{equation}
\begin{equation}
\lambda_{J}=\sqrt{\frac{\pi {c_s}^2}{G \rho_0}}
\end{equation}
\begin{equation}
M_{J}=\frac{4 \pi}{3} \rho_0 \left ( \frac{\lambda_J}{2} \right)
\end{equation}
\begin{equation}
\tau_{cool}=\frac{3}{2} \frac{1.381 \times 10^{-16}~erg~K^{-1} \cdot 80 K}{10^{-26} \frac{erg~cm^3}{s} \cdot 42 cm^{-3}} = 3.956 \times 10^{10} s
\end{equation}
\begin{equation}
\tau_{ff}= \sqrt{\frac{3 \pi}{32 6.674 \times 10^{-11}~m^3~kg^{-1}~s^{-2} \rho}} = 123 s
\end{equation}
(WNM und) WIM Phase: \\
T=8000 K, $n = 0.25 cm^{-3}$, $x=0.68$, $\Lambda_0 = 10^{-22} \frac{erg~cm^3}{s}$\\
$\rho= 22 \frac{kg}{m^3}$, $c_{s}=33 \frac{m}{s}$, $\lambda_{J}= 321 m$, $M_{J}= 321 kg$, $\tau_{cool} = 123 s$ und $\tau_{ff} = 123 s$\\
HIM Phase: \\
$T=4.5 \times 10^{5} K$, $n = 3.5 \times 10^{-3} cm^{-3}$, $x=1.0$, $\Lambda_0 = 10^{-22} \frac{erg~cm^3}{s}$\\
$\rho= 22 \frac{kg}{m^3}$, $c_{s}=33 \frac{m}{s}$, $\lambda_{J}= 321 m$, $M_{J}= 321 kg$, $\tau_{cool} = 123 s$ und $\tau_{ff} = 123 s$\\
