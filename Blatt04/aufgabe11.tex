\section{Aufgabe 11}
\subsection{Welche Elemtarteilchen}
Im Kosmos besteht die Kosmische Strahlung aus Primärteilchen, dass sind Protonen, Alphateilchen, Photonen, Neutrinos, schwere und leichte Atomkerne
Bei der Erdatmosphäre werde dann diese geladene Primärteilchen zu Sekundärteilchen(Pionen, Myonen, Protonen, Neutronen und andere Zerfallsprodukte). Die Sekundärteilchen kann man auch zur Detektierung von kosmischer Strahlung verwenden.
\subsection{Welche Detektoren}
\begin{description}
    \item Waren in Betrieb
	\begin{enumerate}
		\item 1912 - Wulf'schen Strahlenapparat 7 Ballonflüge 
		\item 1993-1996 - High Resolution Fly's Eye Cosmic Ray Detector
		\item KASCADE-Grande (KArlsruhe Shower Core and Array DEtector-Grande)
		\item HEAO1, HEAO2, HEAO3
		\item BESS (Balloon-borne Experiment with Superconducting Spectrometer)
		\item ATIC (Advanced Thin Ionization Calorimeter)
		\item BOOMERanG experiment
		\item TIGER
		\item CREAM
	\end{enumerate}
	\item Sind in Betrieb
	\begin{enumerate}
		\item Chandra
		\item H.E.S.S-Teleskop (ein Tscherenkow-Teleskop)
		\item ALBORZ Observatory
		\item CHICOS
		\item Piere Auger Observatory
		\item Telescop Arra Project
		\item WALTA
		\item PAMELA
		\item ACE (Advanced Composition Explorer)
		\item Voyager 1 und Voyager 2
		\item Cassini-Huygens
	\end{enumerate}
	\item Werden gerade gebaut
	\begin{enumerate}
		\item MARIACHI
		\item Alpha Magnetic Spectrometer
		\item Spaceship Earth
		\item TRACER (cosmic ray detector)
	\end{enumerate}
\end{description}

\subsection{Was wird gemessen}
Geschwindigkeit, Zusammensetzung, Teilchendichte, Massendichte, Teilchenstromdichten und Teilchenenergie.
