
\documentclass[fontsize=12pt,a4paper]{article}

%\usepackage{fullpage}
\usepackage[ngerman]{babel}
\usepackage[latin9]{inputenc}
\usepackage{amsmath}

\addtolength{\voffset}{-50pt}
\addtolength{\textheight}{600pt}

\title{\bfseries Einf�hrung in die Astronomie II (UE)\\
Aufgabe 3}
\author{Camillo Dell'mour, 0628020\\
Rainer Mursch-Radlgruber, 0828445}
\date{\today}

\begin{document}

\maketitle

\subsection*{Ionisation aus dem Balmer-Niveau (\(n=2 \to \infty\))}

Es gilt (Dispersionsrelation):
\[E=\frac{h \cdot c}{\lambda}\]
\(E\dots\) Energie in \(eV\)\\
\(\lambda\dots\) Wellenl�nge in \(m\)\\
\(h\dots\) Plancksches Wirkungsquantum, \(h=4.13566733 \times 10^{-15}~eV~s\)\\
\(c\dots\) Lichtgeschwindigkeit, \(c=2.99792458 \times 10^8~m~s^{-1}\)
\\\\
\begin{tabular}{l l l}
Lyman-Kante (\(n=1 \to \infty\)):&\(\lambda_1=912~\AA\)&\(E_1=13.6~eV\)\\
Ly-alpha (\(n=2 \to 1\)): & \(\lambda_2=1216~\AA\)&\(E_2=\frac{h \cdot c}{\lambda_2}=10.2~eV\)
\end{tabular}
\\\\
\(\implies E=E_1-E_2=3.4~eV\)

\subsection*{Wellenl�nge der Balmer-Kante (\(n=2 \to \infty\))}

\(\lambda=\frac{h \cdot c}{E}=3646.6~\AA\)

\subsection*{Umrechnung (\(eV \to erg \to Joule\))}

\(1~eV=1.602176 \times 10^{-12}~ergs\)\\
\(1~erg=10^{-7}~J\)

\subsection*{Lyman-Kontinuums-Leuchtkraft \(L_{Lyc}\)\\eines \(42000~K\) hei�en Sterns}
Lyman-Kontinuums-Photonenfluss \(log_{10}(N_L)=49.08~s^{-1}\)\\
Energie eines Photons im Mittel \(\overline E=49.08eV\)\\\\
\(
\begin{array}{l l}
\implies&L_{Lyc}=10^{49.08} \cdot 20 \cdot 1.60217 \times 10^{-12}~ergs~s^{-1}\\
&L_{Lyc}=3.8525 \times 10^{38}~ergs~s^{-1}
\end{array}
\)

\subsection*{�ber 4 Mio. Jahre abgegebene Energie\\der Lyman-Kontinuums-Strahlung}
\(t=4 \times 10^6~years=1.261 \times 10^{14}~s\)\\\\
\(
\begin{array}{l l}
\implies&E_{Lyc}=3.8525 \times 10^{38} \cdot 1.261 \times 10^{14}~ergs\\
&E_{Lyc}= 4.858 \times 10^{52}~ergs
\end{array}
\)
\end{document}
